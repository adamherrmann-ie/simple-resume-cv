% !TEX TS-program = xelatex
% !TEX encoding = UTF-8 Unicode
% -*- coding: UTF-8; -*-
% vim: set fenc=utf-8

%%%%%%%%%%%%%%%%%%%%%%%%%%%%%%%%%%%%%%%%%%%%%%%%%%%%%%%%%%%%%%%%%
%% SIMPLE-RESUME-CV
%% <https://github.com/zachscrivena/simple-resume-cv>
%% This is free and unencumbered software released into the
%% public domain; see <http://unlicense.org> for details.
%%%%%%%%%%%%%%%%%%%%%%%%%%%%%%%%%%%%%%%%%%%%%%%%%%%%%%%%%%%%%%%%%

% See "README.md" for instructions on compiling this document.

\documentclass[letterpaper,MMMyyyy,nonstopmode]{simpleresumecv}
% Class options:
% a4paper, letterpaper, nonstopmode, draftmode
% MMMyyyy, ddMMMyyyy, MMMMyyyy, ddMMMMyyyy, yyyyMMdd, yyyyMM, yyyy

%%%%%%%%%%%%%%%%%%%%%%%%%%%%%%%%%%%%%%%%%%%%%%%%%%%%%%%%%%%%%%%%%
%% PREAMBLE.
%%%%%%%%%%%%%%%%%%%%%%%%%%%%%%%%%%%%%%%%%%%%%%%%%%%%%%%%%%%%%%%%%

% CV Info (to be customized).
\newcommand{\CVAuthor}{Adam Herrmann}
\newcommand{\CVTitle}{}
\newcommand{\CVNote}{CV compiled on {\today}}
\newcommand{\CVWebpage}{http://www.adamherrmann.ie}

% PDF settings and properties.
\hypersetup{
pdftitle={\CVTitle},
pdfauthor={\CVAuthor},
pdfsubject={\CVWebpage},
pdfcreator={XeLaTeX},
pdfproducer={},
pdfkeywords={},
unicode=true,
bookmarks=true,
bookmarksopen=true,
pdfstartview=FitH,
pdfpagelayout=OneColumn,
pdfpagemode=UseOutlines,
hidelinks,
breaklinks}

% Shorthand.
\newcommand{\Code}[1]{\mbox{\textbf{#1}}}
\newcommand{\CodeCommand}[1]{\mbox{\textbf{\textbackslash{#1}}}}

\linespread{1.2}

%%%%%%%%%%%%%%%%%%%%%%%%%%%%%%%%%%%%%%%%%%%%%%%%%%%%%%%%%%%%%%%%%
%% ACTUAL DOCUMENT.
%%%%%%%%%%%%%%%%%%%%%%%%%%%%%%%%%%%%%%%%%%%%%%%%%%%%%%%%%%%%%%%%%

\begin{document}

%%%%%%%%%%%%%%%
% TITLE BLOCK %
%%%%%%%%%%%%%%%

\Title{\CVAuthor}

\begin{SubTitle}
Wottonstown, Castlebellingham, Co. Louth, A91 C653, Ireland
\par
\href{mailto:adam@adamherrmann.ie}
{adam@adamherrmann.ie}
\,\SubBulletSymbol\,
+353-86-251-1071
\,\SubBulletSymbol\,LinkedIn:
\href{\CVWebpage}
{\url{\CVWebpage}}
\end{SubTitle}

\begin{Body}

%%%%%%%%%%%%%%%
%% INTERESTS %%
%%%%%%%%%%%%%%%

\Section
{Profile}
{Profile}
{PDF:Profile}

\Entry
Jack of all trades Systems Engineer with a passion for solving hard problems. I am passionate about Model Based Systems Engineering and am an expert in system modelling using Matlab, Simulink, Python, and Modern C++ with SystemC. I have owned several models for various ASIC programmes, from three generations of Intel's NPU Engine to high-speed optical modems for LEO satellites. For the latter, I was involved in all aspects of the product lifecycle, from architecture through to tape-out, owning Calibration Algorithm Development (Embedded Firmware), Matlab/Simulink Modelling (HDL Coder/Verifier) and Cyberhardening Strategy. I am a certified Project Management Professional (PMP) and have two patents pending with the US Patents Office relating to novel calibration algorithm implementations for optical modems.

%%%%%%%%%%%%%%%
%% PROGRAMMING LANGUAGES %%
%%%%%%%%%%%%%%%

\Section
{Technical Skills}
{Technical Skills}
{PDF:Technical Skills}

\textbf{Programming Languages:} MATLAB, C++, SystemC, Python, Rust.
\Gap
\textbf{Tools/Flows:} MATLAB, Simulink, HDL Coder/Verifier, Embedded Coder, CMake, Visual Studio, Git, Bash, Jira, Model Based Design, Software/Hardware-in-the-Loop.
\Gap
\textbf{Soft Skills:} Agile Scrum, Project Management \textbf{(PMP certified)}, Mentoring, Team Development.

%%%%%%%%%%%%%%%%%%%%%%%%%
%% ENGINEERING EXPERIENCE %%
%%%%%%%%%%%%%%%%%%%%%%%%%

\Section
{Experience}
{Experience}
{PDF:Experience}

\Entry
\href{http://www.altera.com}
{\textbf{ALTERA}},
Remote, Ireland

\Gap
\BulletItem
\textbf{Systems Engineer}\hfill
\DatestampYMD{2022}{06}{15} --
Present
\begin{Detail}
\Gap
\SubBulletItem
Leading the Matlab/Simulink System Modelling team, modelling all aspects of our reconfigurable system including Optical (MZM, 90 hybrid, optical wave-guides and switches), Electrical, Channel, Digital, and Firmware.
\SubBulletItem
Championing the Shift-Left, Automate-Right philosophy with Matlab Toolboxes like Embedded Coder, HDL Coder, and HDL Verifier. Represented Altera at Mathworks Advisory Board 2025.
\SubBulletItem
Leading DSP calibration algorithm architecture, design, and firmware implementation to ensure system performance and reliability with varying environmental and operational impairments.
\SubBulletItem
Conducting FPGA-based cybersecurity investigations to enhance system resilience in mission-critical applications.
\SubBulletItem
Contributing to international working group meetings to define interoperability standards.
\SubBulletItem
Presenting progress and results to key stakeholders, including government agencies and research institutions.\smallskip \\
These efforts have contributed to securing multimillion-dollar funding for advanced communication programs and driving innovation in next-generation satellite communications and beyond.
\end{Detail}

\Entry
\href{http://www.intel.com}
{\textbf{Intel Corporation}},
Leixlip, Co. Kildare, Ireland

\Gap
\BulletItem
\textbf{C++ Modelling Engineer} (Deep Learning Inference Architecture)
\hfill
\DatestampYMD{2017}{06}{15} --
\DatestampYMD{2022}{06}{15}
\begin{Detail}
\Gap
\SubBulletItem
Team Lead/Product Owner after two years: Leading the Agile Transformation by being the first modelling team to use a Scrum based approach for task planning and management. Actively giving training to other modelling teams on Scrum practices and Jira.
\SubBulletItem
Actively leading and developing modern and performant C++/SystemC models for integration in both RTL verification and software virtual platform environments. These models allow embedded software teams to test their software prior to FPGA deployed RTL being available.
\SubBulletItem
Co-authored a SNUG (Synopsys Users Group) Conference Paper detailing a novel use of the TLM-2.0 (Transaction Level Modelling) framework for RTL verification. Available here: \href{https://www.synopsys.com/news/pubs/snug/2020/europe/jordan-herrmann-paper.pdf}{https://www.synopsys.com/news/pubs/snug/2020/europe/jordan-herrmann-paper.pdf}
\SubBulletItem
Performed architectural performance exploration and in-depth comparative analysis with alternative designs using C++/SystemC models. These lower level models are developed either from a high level architectural specification or a Python/MATLAB model.
\SubBulletItem
Headed the adoption of a common CMake/VSCode/CI development system across the modelling team to enable a greater level of cross team collaboration.
\end{Detail}

\pagebreak

\BulletItem
\textbf{Design Automcation Engineer (Internship)}
\hfill
\DatestampYMD{2016}{01}{15} --
\DatestampYMD{2016}{08}{15}
\begin{Detail}
\Gap
\SubBulletItem
Provided 24/7 infrastructure support (engineering compute and CAD tools) for a team of approx. 150 engineers, during a critical tape-in period.

\SubBulletItem
Provided training classes on Unix and Git to new-hires after four months.

\end{Detail}

%%%%%%%%%%%%%%%
%% EDUCATION %%
%%%%%%%%%%%%%%%

\Section
{Education}
{Education}
{PDF:Education}

\Entry
\textbf{Professional Development}

\Gap
\BulletItem
Space Programme Management
\hfill
\DatestampYMD{2025}{05}{09}
\BulletItem
Winning European Space Agency Proposals
\hfill
\DatestampYMD{2024}{11}{09}
\BulletItem
Project Management (PMP) - UCD Professional Academy
\hfill
\DatestampYMD{2021}{07}{01} --
\DatestampYMD{2021}{10}{01}
\BulletItem
Advanced Real-Time C++ - Feabhas
\hfill
\DatestampYMD{2020}{04}{01}
\BulletItem
Practical Deep Learning - Doulos
\hfill
\DatestampYMD{2019}{01}{01}

\Gap
\Entry
\href{http://www.ucd.ie}
{\textbf{University College Dublin}},
Belfield, Dublin, Ireland

\Gap
\BulletItem
\textbf{
M.E. in
\href{https://www.ucd.ie/eacollege/}
{Electronic \& Computer Engineering}}
\hfill
\DatestampYMD{2015}{09}{15} --
\DatestampYMD{2017}{05}{15}
\begin{Detail}
\Gap
\SubBulletItem
First Class Honours - S3 Group EGA Gold Medal for coming first in class.
\SubBulletItem
Relevant Modules: Software Engineering (A+), Modelling & Simulation (A+), Distributed Optimisation \& Control over Networks (A+), Unix Programming (A+).
\Gap
\SubBulletItem
\textbf{Final Year Project} (Grade Achieved: A):
A Situational Aware In-Car System to Protect Cyclists.\\
Developed a smart in-car system to detect and protect cyclists from both collisions and pollution caused by motor-cars, utilising the energy actuation abilities of hybrid vehicles. Solution comprised of a mobile client application to alert the driver of the cyclist's presence and to collect real-time driving data; as well as a Java/Python server placed in the rear of a Toyota Prius to handle data from the antenna at the rear of the car and to simulate the presence of other cars in the area to generate a pollution model. Data driven Model Based Design was used to develop in-car system to analyse antenna data. Video showing Hardware-in-the-Loop demo of system available here: \href{http://ucd.adamherrmann.ie/me_thesis/videos/Ch8_HardwareInTheLoopDemo.mp4}
{http://ucd.adamherrmann.ie/me\_thesis}. Paper published in IEEE Transactions on Intelligent Transportation Systems \href{https://ieeexplore.ieee.org/document/8293816/}{https://ieeexplore.ieee.org/document/8293816}.
\end{Detail}

\Gap
\BulletItem
\textbf{
B.Sc. in
\href{https://www.ucd.ie/eacollege/}
{Engineering Science (Electronic)}}
\hfill
\DatestampYMD{2012}{09}{15} --
\DatestampYMD{2015}{05}{15}
\begin{Detail}
\Gap
\SubBulletItem
First Class Honours.
\SubBulletItem
UCD Entrance Scholar for achieving 580/625 Points in the Leaving Certificate (top 3\% nationally).
\SubBulletItem
Highest Ranked Robot in the UCD RoboRugby Team Competition (Part of the Robotics Design Project module).

\end{Detail}

%%%%%%%%%%%%%%%
%% PUBLICATIONS %%
%%%%%%%%%%%%%%%

\Section
{Publications}
{Publications}
{PDF:Publications}

\Entry
\textbf{Matlab EXPO Nov 2025} \\Model Once, Deploy Anywhere: HDL and Embedded Coder Applied to Free Space Optics

\Gap
\textbf{SNUG Europe May 2020} \\Enabling Faster Debug by Seamlessly Integrating SystemC Models into a UVM TB
 
\Gap
\textbf{IEEE Transactions on Intelligent Transportation Systems Feb 2018} \\A New Take on Protecting Cyclists in Smart Cities

%%%%%%%%%%%%%%%
%% PATENTS %%
%%%%%%%%%%%%%%%

\Section
{Patents}
{Patents}
{PDF:Patents}

\Entry
\textbf{US20250219735A1 - Filed Dec 27, 2023} \\In-field droop measurement and compensation for coherent optical transceiver

\Gap
\textbf{US20240146416A1 - Filed Nov 2, 2022} \\Method and apparatus for enhancement of common mode rejection on coherent optic receivers

%%%%%%%%%%%%%%%
%% INTERESTS %%
%%%%%%%%%%%%%%%

\Section
{Interests}
{Interests}
{PDF:Interests}

\Entry
Aviation, Swimming, and Stand-up/Improv.

\end{Body}

\end{document}
